\section{Data Structures}

The project will mainly rely on the following data structures:

\subsection{Pandas Dataframe}

All bars fetched from OpenStreetMap will be stored in a \texttt{pandas.DataFrame}
with columns such as:

\begin{table}[H]
\centering
\caption{Structure of the bar Dataframe}
\label{tab:df}
\begin{tabular}{llll}
\toprule
\textbf{Column} & \textbf{Type} & \textbf{Description} \\ 
\midrule
name & string & Venue name \\
lat, lon & float & Coordinates of the bar \\
tags & dict & OSM metadata (amenity, cuisine, music etc.) \\
opening\_hours & string / parsed object & Extracted from tags \\
dist\_user & float & Distance to user location (meters) \\
type\_derived & string & Normalised bar category \\
score\_total & float & Final overall recommendation score \\
\bottomrule
\end{tabular}
\end{table}



This structure makes it easy to filter bars, compute scores and export to CSV
if needed.

% \subsection{Python Dictionaries}

% Dictionaries will be used to store user preferences and tag mappings, for example:
% \begin{itemize}[leftmargin=*]
%     \item a preference dictionary such as\\
%           \texttt{\{"style": "student bar", "music": "jazz"\}}
%     \item mappings from raw OSM tags to simplified categories (e.g., mapping
%           several tags to the label \enquote{sports bar})
% \end{itemize}

% \subsection{Graph Structure (NetworkX)}

% For route calculation, I will build a weighted graph where:

% \begin{itemize}[leftmargin=*]
%     \item nodes represent bars (and optionally the starting location),
%     \item edges represent possible walking paths between bars,
%     \item edge weights store the distance between locations.
% \end{itemize}

% This graph is then used with NetworkX functions or heuristics to compute a
% reasonable pub crawl order.

\subsection{Dataframe of suitable bars}

Intermediate results, such as the list of bars matching user preferences, will
be stored in panda dataframes whit all the necessary informations before they
are passed into the graph-based route computation.